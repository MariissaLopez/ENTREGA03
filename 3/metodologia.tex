

\chapter{METODOLOGÍA DE LA INVESTIGACIÓN}

\section{Diseño de la investigación}

\subsection{Tipo de la investigación}

El tipo de investigacion para este presente trabajo es experimental, puesto que se  analizan imágenes y se lleva a cabo un diseño de implementación y evaluación de una aplicacion por vision por computadora que monitorea el consumo de carbohidratos de los alimentos de estudiantes universitarios. También se realizarán pruebas para medir la efectividad de la aplicación y notar si es de utilidad para la prevensión del síndrome meabólico.

\subsection{Enfoque de la investigacion}

El enfoque para este presente trabajo es cuantitativo, debido a que se realizará una representación de pixeles por cada imagen  presentada en la base de datos, seguido de un redimensionamiento y representación numérica. También los instrumentos a usar se harán por aproximaciones que serán medidas, tales como el uso del precisión y sensibilidad. 

\subsection{Poblacion}

La población es representada por las imágenes de alimentos que son consumidos por estudiantes universitarios, como platos de comida peruanos e internacionales, imágenes de postres simples y comidas rápidas extraídas de la plataforma Kaggle. Las imágenes deben estar en buena calidad y no tener ruido, es decir, no estar cargadas de otros objetos que no sean los alimentos requeridos, tampoco deben de presentarse borrosas. 

\subsection{Muestra}

La muestra es representada por mas de mil imagenes de platos de comida, puede ser del tipo platos de comidas peruanas e internacionales, postres y comidas rápidas, 

\subsection{Operacionalización de Variables}

En la Tabla \ref{3:table1} se presenta la operacionalización de variables para la investigación.

\begin{table}[h!]
  \centering
  \caption{Variables e Indicadores}
  \label{3:table1}
  \begin{tabular}{|p{0.5\linewidth}|p{0.4\linewidth}|}
    \hline
    \textbf{Variables} & \textbf{Indicadores} \\
    \hline
    Dependiente: Prevención del síndrome metabólico en estudiantes universitarios.
    
    \thinspace
    
    Independiente: Aplicación para monitorear carbohidratos aplicando visión por computadora.
    & \vspace{1mm} Precisión (Accuracy): $\frac{TP + TN}{TP + TN + FP + FN}$ \\
    & Recall (Sensibilidad): $\frac{TP}{TP + FN}$ \\
    & F1-score: $2 \times \frac{\text{Precision} \times \text{Recall}}{\text{Precision} + \text{Recall}}$ \\
    & Área bajo la curva ROC \\
    \hline
  \end{tabular}
\end{table}

El trabajo busca prevenir el síndrome metabolico  a través de una aplicacion de visión por computadora que monitorea el consumo de carbohidratos. 